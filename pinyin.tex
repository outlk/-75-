% !TeX program = xelatex
\documentclass[12pt]{ctexart}

\usepackage[
    a4paper,
    hmargin={2cm},
    vmargin={2.5cm}
]{geometry}

\setmainfont{Times New Roman}
\setCJKmainfont[AutoFakeBold]{华文中宋}

\usepackage{tocloft}
\setlength\cftbeforesecskip{0em}

\usepackage{booktabs}
\usepackage{longtable}

\ctexset{
    section={
        name={第,首},
        number=\arabic{section},
        format=\zihao{-1} \centering \bfseries
        },
}

\usepackage{xpinyin,xcolor,graphicx}
\xpinyinsetup{ratio={0.7},vsep={1em},hsep={1em plus .5em},pysep={~~~},font={\normalfont},format={\color{black}},multiple={\color{red}}}
% \newcounter{tnumber}[section]
% \newcommand{\itemm}{\stepcounter{tnumber} \makebox[3.5em][s]{第\arabic{tnumber}课}}
% \newcommand{\itemn}{\stepcounter{tnumber} \makebox[1.5em][s]{\arabic{tnumber}.}}
\usepackage{setspace}%使用间距宏包

\renewcommand{\CJKglue}{\hskip 1cm plus 0.08\baselineskip} 
\renewcommand{\baselinestretch}{1.7}

\newcommand\shuangxi[1]{\begin{spacing}{1.5} \zihao{-3} \noindent {\bfseries【赏析】} #1 \end{spacing} }

\title{\zihao{0} \bfseries 小学生必背古诗75首}
\author{\zihao{2} \bfseries 义务教育语文课程标准(2022年版)}
\date{\zihao{3}2022年10月15日}

\begin{document}

\pagestyle{plain}

\pagenumbering{Roman}

\maketitle

\tableofcontents

\cleardoublepage

\pagenumbering{arabic}
\setcounter{page}{1}

\zihao{2}

\section{江南}
\vspace{-1.0em} \section*{汉\xpinyin{乐}{yuè}府}
\begin{center}
    江南可采莲,莲叶何田田,鱼戏莲叶间。\\
    鱼戏莲叶东,鱼戏莲叶西,\\   
    鱼戏莲叶南,鱼戏莲叶北。\\
\end{center}
\shuangxi{青年男女在江边采莲,茂盛的莲叶在江中生长,有些游鱼在莲叶间穿梭往还。鱼儿一会儿游向东,一会儿游向西,一会儿游向南,一会儿游向北,无拘无束,怡然自得。}

\section{长歌行}
\vspace{-1.0em} \section*{汉乐府}
\begin{center}
    青青园中\xpinyin{葵}{kuí},\xpinyin{朝}{zhāo}露待日\xpinyin{晞}{xī}。\\
    阳春布德泽,万物生光辉。\\
    常恐秋节至,\xpinyin{焜}{kūn}黄\xpinyin{华}{huā}叶衰。\\
    百川东到海,何时复西归?\\
    \xpinyin{少}{shào}壮不努力,老大\xpinyin{徒}{tú}伤悲。\\
\end{center}
\shuangxi{园中的葵菜都郁郁葱葱,晶莹的朝露阳光下飞升。春天把希望洒满了大地,万物都呈现出一派繁荣。常恐那肃杀的秋天来到,树叶儿黄落百草也凋零。百川奔腾着东流到大海,何时才能重新返回西境?少年人如果不及时努力,到老来只能是悔恨一生。}


\section[敕勒歌]{\xpinyin{敕}{chì}\xpinyin{勒}{lè}歌}
\vspace{-1.0em} \section*{北朝民歌}
\begin{center}
    敕勒川,阴山下。\\
    天似\xpinyin{穹}{qióng}\xpinyin{庐}{lú},笼盖四\xpinyin{野}{yě}。\\
    天苍苍,野茫茫,风吹草低\xpinyin{见}{xiàn}牛羊。\\
\end{center}
\shuangxi{敕勒人生活的原野在阴山脚下,这里的天幕象毡帐篷一样笼罩着辽阔的大地。苍天浩渺无边,草原茫茫无际,每当大风儿吹来草儿低伏的时候,放牧的牛羊就显现出来。}


\section{咏鹅}
\vspace{-1.0em} \section*{唐 $\cdot$ 骆宾王}
\begin{center}
    鹅,鹅,鹅,\xpinyin{曲}{qū}项向天歌。\\
    白毛浮绿水,红掌拨清波。\\
\end{center}
\shuangxi{“呷哦,呷哦”——多好看的鹅!曲着脖子仰起头,对着青天唱赞歌。雪白的毛,碧绿的水,一对鲜红的脚掌,轻轻地,拨弄起两行清清的水波。}


\section{风}
\vspace{-1.0em} \section*{唐 $\cdot$ 李\xpinyin{峤}{qiáo}}
\begin{center}
    解落三秋叶,能开二月花。\\
    过江千尺浪,入竹万\xpinyin{竿}{gān}\xpinyin{斜}{xié}。\\
\end{center}
\shuangxi{风,能使晚秋的树叶脱落,能催开早春二月的鲜花,它经过江河时能掀起千尺巨浪, 刮进竹林时可把万棵翠竹吹得歪歪斜斜。}


\section{咏柳}
\vspace{-1.0em} \section*{唐 $\cdot$ 贺知章}
\begin{center}
    碧玉妆成一树高,万条垂下绿丝\xpinyin{绦}{tāo}。\\
    不知细叶谁裁出,二月春风似剪刀。\\
\end{center}
\shuangxi{如玉雕一样的新柳碧玉婆娑,无数柔嫩的柳条挂下来象丝带一样。细长的柳叶是谁剪出来的呢?原来二月的春风就是一把神奇的剪刀。}


\section{回乡偶书}
\vspace{-1.0em} \section*{唐 $\cdot$ 贺知章}
\begin{center}
    少小离家老大回,乡音无改\xpinyin{鬓}{bìn}毛\xpinyin{衰}{shuāi}。\\
    儿童相见不相识,笑问客从何处来。\\
\end{center}
\shuangxi{我从小离开家乡年老才返回,我的乡音没变头发却已稀疏。家乡的儿童看我是远方的客人,他们笑着问我从何处来。}


\section{凉州词}
\vspace{-1.0em} \section*{唐 $\cdot$ 王之\xpinyin{涣}{huàn}}
\begin{center}
    黄河远上白云间,一片孤城万\xpinyin{仞}{rèn}山。\\
    \xpinyin{羌}{qiāng}笛何须怨杨柳,春风不度玉门关。\\
\end{center}
\shuangxi{黄河远远流淌直飞上白云端,孤零零一座边城屹立在崇山峻岭间。兵士们何必吹着羌笛奏起衰怨的《折杨柳》,温情的春风难以吹过玉门边关。}


\section[登鹳雀楼]{登\xpinyin{鹳}{guàn}雀楼}
\vspace{-1.0em} \section*{唐 $\cdot$ 王之涣}
\begin{center}
    白日依山尽,黄河入海流。\\
    欲穷千里目,更上一层楼。\\
\end{center}
\shuangxi{太阳贴着山落下去了,黄河水滚滚流向大海。想要看到千里之外更壮观的景色,那 就得再登上一层楼。}


\section{春晓}
\vspace{-1.0em} \section*{唐 $\cdot$ 孟浩然}
\begin{center}
    春眠不觉晓,处处闻啼鸟。\\
    夜来风雨声,花落知多少。\\
\end{center}
\shuangxi{春夜里贪睡眠不知不觉天已破晓,醒来时四处已传来鸟儿的啼叫。忽记起昨夜听到刮风下雨的声音,谁知道满枝盛开的花朵凋零多少。}


\section{凉州词}
\vspace{-1.0em} \section*{唐 $\cdot$ 王\xpinyin{翰}{hàn}}
\begin{center}
    葡萄美酒夜光杯,欲饮\xpinyin{琵}{pí}\xpinyin{琶}{pa}马上催。\\
    醉卧沙场君莫笑,古来征战几人回?\\
\end{center}
\shuangxi{精美的酒杯中斟满了葡萄酒,战士们正在准备举杯开怀畅饮,琵琶声从远处传来,催促他们上马出发。战士们喝醉了躺在战场上。您可别见笑,自古以来当兵打仗,有几个能从战场上平平安安地回来啊!}


\section{出塞}
\vspace{-1.0em} \section*{唐 $\cdot$ 王昌龄}
\begin{center}
    秦时明月汉时关,万里长征人未还。\\
    但使龙城飞\xpinyin{将}{jiàng}在,不教胡马度阴山。\\
\end{center}
\shuangxi{还是秦时的明月汉时的边关,远离家乡的将士仍未回还。要是现在有像李广那样的统帅,绝不会让匈奴的军队侵扰阴山。}


\section[芙蓉楼送辛渐]{\xpinyin{芙}{fú}\xpinyin{蓉}{róng}楼送辛渐}
\vspace{-1.0em} \section*{唐 $\cdot$ 王昌龄}
\begin{center}
    寒雨连江夜入吴,平明送客楚山孤。\\
    洛阳亲友如相问,一片冰心在玉壶。\\
\end{center}
\shuangxi{冒着深秋冷雨,连夜到吴地,第二天清晨客人就要离去。客人走后,独留江南的我就像楚山一样孤单寂寞了。你回到洛阳,若是亲人们问起我,就说我的心像玉壶里的冰块一样晶莹透明,官场得失置之度外。}


\section[鹿柴]{鹿\xpinyin{柴}{zhài}}
\vspace{-1.0em} \section*{唐 $\cdot$ 王维}
\begin{center}
    空山不见人,但闻人语响。\\
    返\xpinyin{景}{jǐng}入深林,复照青\xpinyin{苔}{tái}上。\\   
\end{center}
\shuangxi{空寂的山中看不见人影,只是偶尔听到来自林中的说话声。落日的余辉返照射入林海深处,又透过密林映照在幽暗的青苔上。}


\section{送元二使安西}
\vspace{-1.0em} \section*{唐 $\cdot$ 王维}
\begin{center}
    \xpinyin{渭}{wèi}城\xpinyin{朝}{zhāo}雨\xpinyin{浥}{yì}轻尘,客\xpinyin{舍}{shè}青青柳色新。\\
    劝君更尽一杯酒,西出阳关无故人。\\
\end{center}
\shuangxi{清晨一场细雨使渭城空气格外清闲,旅舍那么明净,柳枝也像梳洗了一番。请您干了这杯酒吧,向西出了阳关,再也见不到老朋友了。}


\section{九月九日忆山东兄弟}
\vspace{-1.0em} \section*{唐 $\cdot$ 王维}
\begin{center}
    独在异乡为异客,每逢佳节倍思亲。\\
    遥知兄弟登高处,遍插\xpinyin{茱}{zhū}\xpinyin{萸} {yú}少一人。\\ 
\end{center}
\shuangxi{独自一个人在他乡就是那里的外来人了,每次到了节日的时候更加思念亲人。在那遥远的家乡亲人们登高望远的时候,他们头插茱萸偏偏只少了我一个人吧!}


\section{静夜思}
\vspace{-1.0em} \section*{唐 $\cdot$ 李白}
\begin{center}
    床前明月光,疑是地上霜。\\
    举头望明月,低头思故乡。\\
\end{center}
\shuangxi{静静的夜晚,床前被明月的光辉照得一片洁白,几乎使人以为是地上铺了一层霜。仰头看看明月,不由得低头深深怀念遥远的家乡。}


\section{古朗月行}
\vspace{-1.0em} \section*{唐 $\cdot$ 李白}
\begin{center}
    小时不识月,呼作白玉盘。\\
    又疑\xpinyin{瑶}{yáo}台镜,飞在青云端。\\
\end{center}
\shuangxi{小时候不认识月亮,把明月叫作白玉盘。又怀疑是瑶台仙镜,飞在夜空云彩中间。}


\section{望庐山瀑布}
\vspace{-1.0em} \section*{唐 $\cdot$ 李白}
\begin{center}
    日照香炉生紫烟,遥看瀑布挂前川。\\
    飞流直下三千尺,疑是银河落九天。\\
\end{center}
\shuangxi{太阳光照在香炉峰瀑布上,四周都升起紫色的烟雾,远远望去,瀑布像长河一般的垂挂下来;飞奔而下的水流,似乎有三千尺那么长,就好像银河从天上最高处直往下落一样。}


\section[赠汪伦]{赠\xpinyin{汪}{wāng}伦}
\vspace{-1.0em} \section*{唐 $\cdot$ 李白}
\begin{center}
    李白乘舟将欲行,忽闻岸上踏歌声。\\
    桃花潭水深千尺,不及汪伦送我情。\\
\end{center}
\shuangxi{李白乘船将要起程,忽然听到岸上有人手拉着手,一边唱歌一边踏着节拍走来。即使桃花潭水千尺深,也比不上汪伦对我的深厚情谊。}


\section{黄鹤楼送孟浩然之广陵}
\vspace{-1.0em} \section*{唐 $\cdot$ 李白}
\begin{center}
    故人西辞黄鹤楼,烟花三月下扬州。\\
    孤帆远影碧空尽,唯见长江天际流。\\
\end{center}
\shuangxi{老朋友孟浩然辞别黄鹤楼到东边的扬州去了。这正是繁花似锦柳绿如烟的暮春三月。我望着他乘坐的那只小船越走越远,最后连那片白帆的影子也在水天相接的地方消失了。这时候只有滚滚的长江水向遥远的天边流去。}


\section{早发白帝城}
\vspace{-1.0em} \section*{唐 $\cdot$ 李白}
\begin{center}
    \xpinyin{朝}{zhāo}辞白帝彩云间,千里江陵一日\xpinyin{还}{huán}。\\
    两岸猿声\xpinyin{啼}{tí}不住,轻舟已过万\xpinyin{重}{chóng}山。\\
\end{center}
\shuangxi{清晨,我告别高入云霄的白帝城;江陵远在千里,船行只需一日时间。两岸猿声,还在耳边不停地啼叫;不知不觉,轻舟已穿过万重青山。}


\section{望天门山}
\vspace{-1.0em} \section*{唐 $\cdot$ 李白}
\begin{center}
    天门中断楚江开,碧水东流至此回。\\
    两岸青山相对出,孤帆一片日边来。\\
\end{center}
\shuangxi{天门山从中间断开,长江畅通奔流,碧绿的江水向东流到这里回旋。东西两岸的东梁山和西梁山夹江对峙耸出来,有一只挂着帆的小船从太阳那边驶来。}


\section{别董大}
\vspace{-1.0em} \section*{唐 $\cdot$ 高适}
\begin{center}
    千里黄云白日\xpinyin{曛}{xūn},北风吹雁雪纷纷。\\
    莫愁前路无知己,天下谁人不识君。\\
\end{center}
\shuangxi{旷野茫茫,西沉的夕阳染黄了天边的乌云,风吹雁鸣,声音凄凉,大雪飘落。不要担心旅途中没有知心朋友,天下有谁不认识您呢?}
    

\section{绝句}
\vspace{-1.0em} \section*{唐 $\cdot$ 杜甫}
\begin{center}
    两个黄鹂鸣翠柳,一行白\xpinyin{鹭}{lù}上青天。\\
    窗含西岭千秋雪,门泊东吴万里船。\\
\end{center}
\shuangxi{两只黄鹂在翠绿的柳枝间鸣叫,一行白鹭在晴朗的蓝天飞翔。透过窗子望见积雪经年的西山雪岭,门外停泊着要到东吴远行的船只。}


\section{春夜喜雨}
\vspace{-1.0em} \section*{唐 $\cdot$ 杜甫}
\begin{center}
    好雨知时节,当春乃发生。\\
    随风\xpinyin{潜}{qián}入夜,润物细无声。\\
    野\xpinyin{径}{jìng}云俱黑,江船火独明。\\
    晓看红湿处,花\xpinyin{重}{zhòng}锦官城。\\
\end{center}
\shuangxi{好雨像适应了季节变化,到了春天就降临。伴随着春风悄悄地飘洒在夜里,滋润着万物,细微而没有声音。田野里的小路、乌云,全部乌黑,只有江中船上的灯火明亮。到天亮时,再看那红色的湿漉漉的地方,春花沉甸甸的,妆点着锦官城。}
    

\section{绝句}
\vspace{-1.0em} \section*{唐 $\cdot$ 杜甫}
\begin{center}
    迟日江山丽,春风花草香。\\
    泥融飞燕子,沙暖睡\xpinyin{鸳}{yuān}\xpinyin{鸯}{yāng}。\\
\end{center}
\shuangxi{在初春明媚的阳光里,江边是多么美丽。和暖的春风轻轻吹着,草绿了,花开了,散发着阵阵清香。泥土解冻了,燕子忙碌地飞来飞去,衔泥造窝;沙子晒暖了,鸳鸯舒适地睡在沙洲上,成对成双。}


\section[江畔独步寻花]{江\xpinyin{畔}{pàn}独步寻花}
\vspace{-1.0em} \section*{唐 $\cdot$ 杜甫}
\begin{center}
    黄师塔前江水东,春光懒困\xpinyin{倚}{yǐ}微风。\\
    桃花一\xpinyin{簇}{cù}开无主,可爱深红爱浅红。\\
\end{center}
\shuangxi{黄师塔前那一江的碧波春水滚滚向东流,春天给人一种困倦让人想倚着春风小憩的感觉。江畔盛开的那一簇无主的桃花映入眼帘,究竟是爱深红色的还是更爱浅红色的呢?}


\section[枫桥夜泊]{枫桥夜\xpinyin{泊}{bó}}
\vspace{-1.0em} \section*{唐 $\cdot$ 张继}
\begin{center}
    月落乌\xpinyin{啼}{tí}霜满天,江枫渔火对愁眠。\\
    姑苏城外寒山寺,夜半钟声到客船。\\
\end{center}
\shuangxi{月亮落下去了,乌鸦不时地啼叫,茫茫夜色中似乎弥漫着满天的霜华,面对江边隐约的枫树和江中闪烁的渔火,愁绪使我难以入眠。夜半时分,苏州城外的寒山寺的钟声,悠悠然飘荡到了客船。}


\section[滁州西涧]{\xpinyin{滁}{chú}州西\xpinyin{涧}{jiàn}}
\vspace{-1.0em} \section*{唐 $\cdot$ \xpinyin{韦}{wéi}\xpinyin{应}{yìng}物}
\begin{center}
    独怜幽草涧边生,上有黄鹂深树鸣。\\
    春潮带雨晚来急,野渡无人舟自横。\\
\end{center}
\shuangxi{我喜爱生长在涧边的幽草,黄莺在幽深的树丛中啼鸣。春潮夹带着暮雨流的湍急,惟有无人的小船横向江心。}


\section{渔歌子}
\vspace{-1.0em} \section*{唐 $\cdot$ 张志和}
\begin{center}
    西塞山前白鹭飞,桃花流水\xpinyin{鳜}{guì}鱼肥。\\
    青\xpinyin{箬}{ruò}\xpinyin{笠}{lì},绿\xpinyin{蓑}{suō}衣,斜风细雨不须归。\\
\end{center}
\shuangxi{西塞山前,白色的鹭鸶在欢快的翱翔,粉红色的桃花在迎春怒放,清澈碧绿的江水潺潺地流淌,鲜美的鳜鱼在水中自由自在地游动。戴上青箬笠,穿上绿蓑衣,在微风细雨中,悠闲地钓鱼,哪里还用得着回家呢!}


\section{塞下曲}
\vspace{-1.0em} \section*{唐 $\cdot$ 卢\xpinyin{纶}{lún}}
\begin{center}
    月黑雁飞高,\xpinyin{单}{chán}\xpinyin{于}{yú}夜\xpinyin{遁}{dùn}逃。\\
    欲将轻骑逐,大雪满弓刀。\\
\end{center}
\shuangxi{没有月光的黑夜,大雁飞得很高。单于趁着黑夜偷偷地逃跑。将军正要率领轻装 的骑兵,追击逃敌,大雪纷纷,落满了将士们的弓刀。}


\section[游子吟]{游子\xpinyin{吟}{yín}}
\vspace{-1.0em} \section*{唐 $\cdot$ 孟郊}
\begin{center}
    慈母手中线,游子身上衣。\\
    临行密密缝,意恐迟迟归。\\
    谁言寸草心,报得三春\xpinyin{晖}{huī}?\\
\end{center}
\shuangxi{慈母用手中的针线,为远行的儿子赶制身上的衣衫。临行前一针针密密地缝缀,怕的是儿子回来得晚衣服破损。有谁敢说,子女像小草那样微弱的孝心,能够报答得了像春晖普泽的慈母恩情呢?}


\section[早春呈水部张十八员外]{早春呈水部张十八员外}
\vspace{-1.0em} \section*{唐 $\cdot$ 韩愈}
\begin{center}
    天街小雨润如\xpinyin{酥}{sū},草色遥看近却无。\\
    最是一年春好处,绝胜烟柳满皇\xpinyin{都}{dū}。\\
\end{center}
\shuangxi{京城大道上空丝雨纷纷,它像酥油般细密而滋润,远望草色依稀连成一片,近看时却显得稀疏零星。这是一年中最美的季节,远胜过绿柳满城的春末。}


\section[望洞庭]{望洞庭}
\vspace{-1.0em} \section*{唐 $\cdot$ 刘\xpinyin{禹}{yǔ}\xpinyin{锡}{xī}}
\begin{center}
    湖光秋月两相和,潭面无风镜未磨。\\
    遥望洞庭山水翠,白银盘里一青\xpinyin{螺}{luó}。\\
\end{center}
\shuangxi{辽阔的洞庭湖波光月色互相辉映,无风的湖面像一面镜子未曾打磨。遥望那湖上青翠的君山,活像白银盘里放一只青螺。}


\section[浪淘沙]{浪淘沙}
\vspace{-1.0em} \section*{唐 $\cdot$ 刘禹锡}
\begin{center}
    九曲黄河万里沙,浪淘风\xpinyin{簸}{bǒ}自天涯。\\
    如今直上银河去,同到牵牛织女家。\\
\end{center}
\shuangxi{弯弯曲曲的黄河流程万里黄沙,风簸水动波浪淘卷着泥沙从天边流来。今天我要沿黄河之水直上天河,与它一同到达牵牛织女的家门。}


\section[赋得古原草送别]{\xpinyin{赋}{fù}得古原草送别}
\vspace{-1.0em} \section*{唐 $\cdot$ 白居易}
\begin{center}
    离离原上草,一岁一枯荣。\\
    野火烧不尽,春风吹又生。\\
    远芳\xpinyin{侵}{qīn}古道,晴翠接荒城。\\
    又送王孙去,\xpinyin{凄}{qī}凄满别情。\\
\end{center}
\shuangxi{原野上的青草是那样的茂密,一年一度,枯萎了又茂盛。野火也烧不死它,春风吹来它又显出勃勃生机。芳草的香气弥漫在整个古道之上,翠绿的青草连接着那荒远的城镇。又把朋友送走了,青青的草也充满了离别的深情。}


\section{池上}
\vspace{-1.0em} \section*{唐 $\cdot$ 白居易}
\begin{center}
    小娃撑小\xpinyin{艇}{tǐng},偷采白莲回。\\
    不解藏踪迹,浮萍一道开。\\
\end{center}
\shuangxi{一群娃娃撑了小船去偷采白莲玩,他们不晓得掩藏踪迹,浮萍被小船荡开,留下一条长长的水路。}


\section{忆江南}
\vspace{-1.0em} \section*{唐 $\cdot$ 白居易}
\begin{center}
    江南好,风景旧曾\xpinyin{谙}{ān}。\\
    日出江花红胜火,春来江水绿如蓝。\\
    能不忆江南?
\end{center}
\shuangxi{江南风景真美!那秀丽的景色早就熟谙。日出时,江边鲜艳的花儿,比火焰还红,春来时,清澈的江水碧绿如蓝。这般令人陶醉的景色,能不使人忆念?}


\section{小儿垂钓}
\vspace{-1.0em} \section*{唐 $\cdot$ 胡令能}
\begin{center}
    蓬头\xpinyin{稚}{zhì}子学垂\xpinyin{纶}{lún},侧坐莓\xpinyin{苔}{tái}草映身。\\
    路人借问遥招手,怕得鱼惊不\xpinyin{应}{yìng}人。\\
\end{center}
\shuangxi{一个头发蓬乱的小孩子正在学垂钓,侧身坐在青苔上绿草映衬着他的身影。遇到有人问路他老远就招着小手,唯恐怕鱼儿被吓跑他不敢大声应答。} 


\section[悯农(一)]{\xpinyin{悯}{mǐn}农(一)}
\vspace{-1.0em} \section*{唐 $\cdot$ 李\xpinyin{绅}{shēn}}
\begin{center}
    锄禾日当午,汗滴禾下土。\\
    谁知盘中\xpinyin{餐}{cān},粒粒\xpinyin{皆}{jiē}辛苦。\\
\end{center}
\shuangxi{农民在火辣辣的太阳底下锄地,汗水滴滴嗒嗒地掉在禾苗下的土地。可是有谁知道碗里的饭,每一粒都是农民用千辛万苦的劳动换来的。}


\section[悯农(二)]{悯农(二)}
\vspace{-1.0em} \section*{唐 $\cdot$ 李绅}
\begin{center}
    春种一粒\xpinyin{粟}{sù},秋收万颗子。\\
    四海无闲田,农夫\xpinyin{犹}{yóu}饿死。\\
\end{center}
\shuangxi{春天种下一颗种子,秋收时可以收获万颗子粒。普天之下并无闲置荒芜的田地,可是仍有农夫饿死。}


\section{江雪}
\vspace{-1.0em} \section*{唐 $\cdot$ 柳宗元}
\begin{center}
    千山鸟飞绝,万\xpinyin{径}{jìng}人踪灭。\\
    孤舟\xpinyin{蓑}{suō}\xpinyin{笠}{lì}翁,独钓寒江雪。\\
\end{center}
\shuangxi{连绵的群山不见鸟的飞影,所有的路上难寻人的行踪。一条孤单的小船上,有一位披蓑戴笠的渔翁,独自在风雪寒冷的江上钓鱼。}


\section{寻隐者不遇}
\vspace{-1.0em} \section*{唐 $\cdot$ \xpinyin{贾}{jiǎ}岛}
\begin{center}
    松下问童子,言师采药去。\\
    只在此山中,云深不知处。\\
\end{center}
\shuangxi{在松树下,我询问童子,他说师父采药去了。只知道他就在这座山里,然而山高云深,真不知道他在哪里。}


\section{山行}
\vspace{-1.0em} \section*{唐 $\cdot$ 杜牧}
\begin{center}
    远上寒山石径\xpinyin{斜}{xié},白云生处有人家。\\
    停车坐爱枫林晚,霜叶红于二月花。\\
\end{center}
\shuangxi{坐:因为。一条弯弯曲曲的小路蜿蜒伸向山顶,在白云飘浮的地方有几户人家。我停下车来是因为喜欢这枫林的景色,那火红的枫叶比二月的花还要红艳。}


\section{清明}
\vspace{-1.0em} \section*{唐 $\cdot$ 杜牧}
\begin{center}
    清明时节雨纷纷,路上行人欲断\xpinyin{魂}{hún}。\\
    借问酒家何处有,牧童\xpinyin{遥}{yáo}指杏花村。\\
\end{center}
\shuangxi{清明节这天细雨纷纷,路上远行的人好像断魂一样迷乱凄凉。向人询问酒家哪里有,牧童远远地指了指杏花村。}


\section{江南春}
\vspace{-1.0em} \section*{唐 $\cdot$ 杜牧}
\begin{center}
    千里莺啼绿\xpinyin{映}{yìng}红,水村山\xpinyin{郭}{guō}酒旗风。\\
    南朝四百八十寺,多少楼台烟雨中。\\
\end{center}
\shuangxi{千里辽阔的江南莺啼燕语,绿叶映衬红花,处处春意浓,水乡山城,酒店前的小旗迎风轻轻摆动。啊,昔日南朝建造的一座座寺庙,如今都隐现在一片迷茫的烟雨之中。}


\section{蜂}
\vspace{-1.0em} \section*{唐 $\cdot$ 罗隐}
\begin{center}
    不论平地与山尖,无限风光尽被占。\\
    采得百花成蜜后,为谁辛苦为谁甜?\\
\end{center}
\shuangxi{无论平原还是高山,凡是百花盛开的地方,都是蜜蜂的属地。采撷了百花酿成了香甜的蜂蜜,它们究竟为谁辛苦,为谁制造甘甜?}


\section{江上渔者}
\vspace{-1.0em} \section*{北宋 $\cdot$ 范\xpinyin{仲}{zhòng}\xpinyin{淹}{yān}}
\begin{center}
    江上往来人,但爱\xpinyin{鲈}{lú}鱼美。\\
    君看一叶舟,出\xpinyin{没}{mò}风波里。\\
\end{center}
\shuangxi{江上来来往往的游人,只知道喜爱鲈鱼味道的鲜美。可是你们看一看,江里那如同一片落叶似的捕鱼船,在风浪里颠簸,时而涌上浪峰,时而陷入浪谷,多危险哪!}


\section{元日}
\vspace{-1.0em} \section*{北宋 $\cdot$ 王安石}
\begin{center}
    爆竹声中一岁除,春风送暖入\xpinyin{屠}{tú}苏。\\
    千门万户\xpinyin{曈}{tóng}曈日,总把新桃换旧符。\\
\end{center}
\shuangxi{鞭炮声中转眼一年就过去了,屠苏酒温暖着心房,春回大地。初升的太阳照的千家万户都是那么明亮,庆贺新年家家把新的桃符换上。}

\section{泊船瓜洲}
\vspace{-1.0em} \section*{北宋 $\cdot$ 王安石}
\begin{center}
    京口瓜洲一水\xpinyin{间}{jiàn},钟山只隔数\xpinyin{重}{chóng}山。\\
    春风又绿江南岸,明月何时照我还?\\
\end{center}
\shuangxi{京口和瓜洲被长江水隔开了,瓜洲到南京中间只隔几座大山。春风吹来使江南沿岸一下子又变绿了,天上的明月啊,你什么时候才能照着我返回到家乡呢?}


\section{书湖阴先生壁}
\vspace{-1.0em} \section*{北宋 $\cdot$ 王安石}
\begin{center}
    茅檐长扫净无苔,花木成\xpinyin{畦}{qí}手自栽。\\
    一水护田将绿绕,两山排\xpinyin{闼}{tà}送青来。\\
\end{center}
\shuangxi{茅草房庭院经常打扫,洁净得没有一丝青苔。花草树木成行满畦,都是主人亲手栽种。庭院外一条小河护卫着农田,把绿色的田地环绕,两座青山推开门,送来青翠的山色。}


\section{六月二十七日望湖楼醉书}
\vspace{-1.0em} \section*{北宋 $\cdot$ 苏\xpinyin{轼}{shì}}
\begin{center}
    黑云翻墨未\xpinyin{遮}{zhē}山,白雨跳珠乱入船。\\
    卷地风来忽吹散,望湖楼下水如天。\\
\end{center}
\shuangxi{黑云在天空翻滚,还未来得及遮住群山,又大又猛的雨点已倾盆而下,放眼望去,在湖光山色的衬托下,只见一片白而透明的雨色,落到湖面上的雨脚如珠,乱蹦乱跳,犹如万珠迸溅,射到船里来了。忽然吹来一阵风,将黑云骤雨一并吹散,于是呈现在眼前的,是水天合一、一片宁静的景象。}


\section{饮湖上初晴后雨}
\vspace{-1.0em} \section*{北宋 $\cdot$ 苏轼}
\begin{center}
    水光\xpinyin{潋}{liàn}\xpinyin{滟}{yàn}晴方好,山色空蒙雨亦奇。\\
    欲把西湖比西子,淡妆浓抹总相\xpinyin{宜}{yí}。\\
\end{center}
\shuangxi{晴天的西湖,水上波光荡漾,闪烁耀眼,正好展示着那美丽的风貌;雨天的西湖,山中云雾朦胧,缥缥渺渺,又显出别一番奇妙景致。我想,最好把西湖比作西子,空蒙山色是她淡雅的妆饰,潋滟水光是她浓艳的粉脂,不管她怎样打扮,总能很好地烘托出天生丽质和迷人的神韵。}


\section{惠崇春江晚景}
\vspace{-1.0em} \section*{北宋 $\cdot$ 苏轼}
\begin{center}
    竹外桃花三两枝,春江水暖鸭先知。\\
    \xpinyin{蒌}{lóu}\xpinyin{蒿}{hāo}满地芦芽短,正是河\xpinyin{豚}{tún}欲上时。\\
\end{center}
\shuangxi{竹林外两三枝桃花初放,鸭子在水中游戏,它们最先察觉了初春江水的回暖。河滩上已经满是蒌蒿,芦笋也开始抽芽,这些可都是烹调河豚的好佐料,而河豚此时正要逆流而上,从大海回游到江河里来了。}


\section{题西林壁}
\vspace{-1.0em} \section*{北宋 $\cdot$ 苏轼}
\begin{center}
    横看成岭侧成峰,远近高低各不同。\\
    不识庐山真面目,只\xpinyin{缘}{yuán}身在此山中。\\
\end{center}
\shuangxi{正看庐山,高岭横空;侧看庐山,峭拔成峰;远近高低,形象各异。为什么总看不清庐山真面目呢?恐怕只是因为自身在这山中的缘故吧!}


\section{夏日绝句}
\vspace{-1.0em} \section*{宋 $\cdot$ 李清照}
\begin{center}
    生当作人杰,死\xpinyin{亦}{yì}\xpinyin{为}{wéi}鬼雄。\\
    至今思项羽,不肯过江东。\\
\end{center}
\shuangxi{活着应当是人中俊杰,死了也要做鬼中英雄。到今天我们特别怀念项羽,因为他死得悲壮,不肯回江东,屈辱偷生。}


\section[三衢道中]{三\xpinyin{衢}{qú}道中}
\vspace{-1.0em} \section*{南宋 $\cdot$ 曾\xpinyin{几}{jī}}
\begin{center}
    梅子黄时日日晴,小溪泛尽却山行。\\
    绿荫不减来时路,添得黄鹂四五声。\\
\end{center}
\shuangxi{梅子黄透了的时候,天天都是晴朗的好天气,乘小舟沿着小溪而行,走到了小溪的尽头,再改走山路继续前行。山路上苍翠的树,与来的时候一样浓密,深林丛中传来几声黄鹂的欢鸣声,比来时更增添了些幽趣。}


\section{示儿}
\vspace{-1.0em} \section*{南宋 $\cdot$ 陆游}
\begin{center}
    死去元知万事空,但悲不见九州同。\\
    王师北定中原日,家\xpinyin{祭}{jì}无忘告乃翁。\\
\end{center}
\shuangxi{本来就知道人死了就什么都没有了,只是为看不见全中国统一而感到悲伤。宋朝的军队向北方进军,收复中原的时候,家祭时不要忘了把这件事告诉你们的父亲。}


\section{秋夜将晓出篱门迎凉有感}
\vspace{-1.0em} \section*{南宋 $\cdot$ 陆游}
\begin{center}
    三万里河东入海,五千仞岳上\xpinyin{摩}{mó}天。\\
    遗民泪尽胡尘里,南望王师又一年。\\
\end{center}
\shuangxi{浩浩荡荡的黄河水,从天边流来,直向东奔去注入大海。连连绵绵的华山峰,挺立在西北高原,高接云霄。陷入金军统治下的老百姓流尽了悲苦的眼泪。他们张大眼睛向着南边盼望着,盼望着国家的军队能北上抗敌,唉,这样的情景啊,又是一年了。}


\section[四时田园杂兴(选一)]{四时田园杂\xpinyin{兴}{xìng}(选一)}
\vspace{-1.0em} \section*{南宋 $\cdot$ 范成大}
\begin{center}
    \xpinyin{昼}{zhòu}出\xpinyin{耘}{yún}田夜绩麻,村庄儿女各当家。\\
    童孙未解\xpinyin{供}{gòng}耕织,也\xpinyin{傍}{bàng}桑阴学种瓜。\\
\end{center}
\shuangxi{\xpinyin{供}{gòng},从事。白天下田去除草,晚上搓麻线,全村年轻男女都不得闲,各司其事,各管一行。那些孩子们,不会耕也不会织,却也不闲着,也就在茂盛成阴的桑树底下学种瓜。}


\section[四时田园杂兴(选二)]{四时田园杂兴(选二)}
\vspace{-1.0em} \section*{南宋 $\cdot$ 范成大}
\begin{center}
    梅子金黄杏子肥,麦花雪白菜花稀。\\
    日长篱落无人过,惟有蜻蜓蛱蝶飞。\\
\end{center}
\shuangxi{一树树梅子变得金黄,杏子也越长越大了;荞麦花一片雪白,油菜花倒显得稀稀落落。天长了,农民忙着在地里干活,中午也不回家,门前没有人走动;只有蜻蜓和蝴蝶绕着篱笆飞来飞去。}


\section[小池]{小池}
\vspace{-1.0em} \section*{南宋 $\cdot$ 杨万里}
\begin{center}
    泉眼无声惜细流,树阴照水爱晴柔。\\
    小荷才露尖尖角,早有蜻蜓立上头。\\
\end{center}
\shuangxi{一道细流缓缓从泉眼中流出,没有一点声音;池畔的绿树在斜阳的照射下,将树阴投入水中,明暗斑驳,清晰可见。还未到盛夏,荷叶刚刚从水面露出一个尖尖角,一只小小的蜻蜓早已立在它的上头。}


\section{晓出净慈寺送林子方}
\vspace{-1.0em} \section*{南宋 $\cdot$ 杨万里}
\begin{center}
    毕竟西湖六月中,风光不与四时同。\\
    接天莲叶无穷碧,映日荷花别样红。\\
\end{center}
\shuangxi{到底是六月里的西湖,景色和别的季节不一样:看上去碧绿色的荷叶无穷无尽,好像一直连接到天边;荷花映在早晨的阳光里,红得特别鲜艳。}


\section{春日}
\vspace{-1.0em} \section*{南宋 $\cdot$ 朱\xpinyin{熹}{xī}}
\begin{center}
    胜日寻芳\xpinyin{泗}{sì}水滨,无边光景一时新。\\
    等闲识得东风面,万紫千红总是春。\\
\end{center}
\shuangxi{一个阳光灿烂的日子来到泗水河边观赏风景,眼前一眼望不到头的景色让人觉得耳目一新,其实春风吹来得时候人们一下子就能感觉到,因为放眼望去的万紫千红都在告诉你春天来了。}


\section{观书有感}
\vspace{-1.0em} \section*{南宋 $\cdot$ 朱熹}
\begin{center}
    半亩方塘一\xpinyin{鉴}{jiàn}开,天光云影共\xpinyin{徘}{pái}\xpinyin{徊}{huái}。\\
    问渠\xpinyin{那}{nǎ}得清如许,\xpinyin{为}{wèi}有源头活水来。\\
\end{center}
\shuangxi{半亩大的方形池塘像一面镜子一样展现在眼前,天空的光彩和浮云的影子都在镜子中一起移动。要问为什么那方塘的水会这样清澈呢?是因为有那永不枯竭的源头为它源源不断地输送活水啊。}


\section[题临安邸]{题临安\xpinyin{邸}{dǐ}}
\vspace{-1.0em} \section*{南宋 $\cdot$ 林升}
\begin{center}
    山外青山楼外楼,西湖歌舞几时休?\\
    暖风\xpinyin{熏}{xūn}得游人醉,直把杭州作\xpinyin{汴}{biàn}州。\\
\end{center}
\shuangxi{山外有青山楼外有高楼,西湖的歌舞什么时候才能停止?温暖的风把游人熏得仿佛喝醉了一般,简直把杭州当成了汴州。}


\section{游园不值}
\vspace{-1.0em} \section*{南宋 $\cdot$ 叶绍翁}
\begin{center}
    应怜\xpinyin{屐}{jī}齿印苍苔,小叩柴\xpinyin{扉}{fēi}久不开。\\
    春色满园关不住,一枝红杏出墙来。\\
\end{center}
\shuangxi{主人大概担心园内的青苔被木屐踩坏吧,我轻轻的敲着树枝做成的小门却一直没有人来开。但那满园美丽的春色怎么能关得住呢?一枝开得艳红的杏花早就悄悄地探出墙来了。}


\section{乡村四月}
\vspace{-1.0em} \section*{南宋 $\cdot$ 翁\xpinyin{卷}{juǎn}}
\begin{center}
    绿满山原白满川,子规声里雨如烟。\\
    乡村四月闲人少,才了蚕桑又插田。\\
\end{center}
\shuangxi{山地和平原上草木茂盛,到处绿油油的,河水涨得满满的,映着天光,呈现一片白色。在如烟似雾的蒙蒙细雨中,不时传来几声布谷鸟的啼鸣。繁忙的四月,家家户户都在忙碌不停,人们刚采来桑叶喂过蚕儿,就又忙着下田插稻秧。}


\section{墨梅}
\vspace{-1.0em} \section*{元 $\cdot$ 王\xpinyin{冕}{miǎn}}
\begin{center}
    \xpinyin{吾}{wú}家洗\xpinyin{砚}{yàn}池头树,个个花开淡墨\xpinyin{痕}{hén}。\\
    不要人夸好颜色,只流清气满\xpinyin{乾}{qián}\xpinyin{坤}{kūn}。\\
\end{center}
\shuangxi{我家小池边的梅树,花朵盛开,朵朵梅花都像是用淡淡的墨水点染而成的。它不想用鲜艳的色彩去吸引人,讨好人,求得人们的夸奖,只愿散发一股清香,让它留在天地之间。}


\section[石灰吟]{石灰\xpinyin{吟}{yín}}
\vspace{-1.0em} \section*{明 $\cdot$ 于谦}
\begin{center}
    千锤万凿出深山,烈火\xpinyin{焚}{fén}烧若等闲。\\
    粉骨碎身浑不怕,要留清白在人间。\\
\end{center}
\shuangxi{经过千万次的锤打才从深山里开采出来,把熊熊烈火的焚烧当做很平常的一件事。只要能把一片清白长留人间,就算粉身碎骨也无所畏惧。}


\section{竹石}
\vspace{-1.0em} \section*{清 $\cdot$ 郑\xpinyin{燮}{xiè}}
\begin{center}
    咬定青山不放松,立根原在破岩中。\\
    千磨万击还坚\xpinyin{劲}{jìng},任\xpinyin{尔}{ěr}东西南北风。\\
\end{center}
\shuangxi{竹子把根深深地扎进破岩中,咬着青山毫不松口。竹石无论受到多大的折磨打击,仍然是那样坚定强劲,任凭来自东西南北的狂风猛刮。}


\section{所见}
\vspace{-1.0em} \section*{清 $\cdot$ 袁\xpinyin{枚}{méi}}
\begin{center}
    牧童骑黄牛,歌声振林\xpinyin{樾}{yuè}。\\
    意欲捕鸣蝉,忽然闭口立。\\
\end{center}
\shuangxi{牧童骑着黄牛在树林中高声歌唱,响亮的歌声在茂密的树林里振荡。忽然听到树上知了的叫声,牧童想去捉住知了,立即停止唱歌,跳下牛背。}


\section[己亥杂诗]{\xpinyin{己}{jǐ}\xpinyin{亥}{hài}杂诗}
\vspace{-1.0em} \section*{清 $\cdot$ \xpinyin{龚}{gōng}自珍}
\begin{center}
    九州生气\xpinyin{恃}{shì}风雷,万马齐\xpinyin{喑}{yīn}究可哀。\\
    我劝天公重\xpinyin{抖}{dǒu}\xpinyin{擞}{sǒu},不拘一格\xpinyin{降}{jiàng}人才。\\
\end{center}
\shuangxi{中华的生机倚仗变革的风雷,这般死气沉沉确实令人悲哀。我劝皇上重新振作精神,不受常规束缚降生各种人才。}


\section{村居}
\vspace{-1.0em} \section*{清 $\cdot$ 高\xpinyin{鼎}{dǐng}}
\begin{center}
    草\xpinyin{长}{zhǎng}莺飞二月天,拂\xpinyin{堤}{dī}杨柳醉春烟。\\
    儿童散学归来早,忙趁东风放纸\xpinyin{鸢}{yuān}。\\
\end{center}
\shuangxi{绿草茂盛,黄莺飞舞,正是二月早春,轻拂堤岸的杨柳沉醉在烟雾之中。乡间的孩子们放学回来得很早,一个个借着东风愉快地放起了风筝。}

\clearpage


\end{document}
